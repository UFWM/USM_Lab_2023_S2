\documentclass[12pt,twoside]{article}

\begin{document}
	\title{\textbf{Optimization and visualization of the planning of a three-node electrical network during a day}}
	\author{\begin{tabular}{rl}
			\textbf{Professor:} & Julio Deride.\\
			\textbf{Specialist:} & Nicol\'as Hern\'andez. \\
			\textbf{Students:} & Vicente Moreno, \\
			& Martina Blanco.
	\end{tabular}}
	\date{}
	\maketitle

	\textbf{Abstract.} In this paper we study the production optimization problem of a three-node electrical network on the course of a day. Each node has different productions and demands to satisfy depending on time, with different technologies contributing to production on all nodes. The three-node approach was inspired by the Chilean electrical network, where we can distinguish the North, South and Center areas as the main producers and consumers of electricity. By studying the mathematical equation describing this problem, we managed to apply a fast solving algorithm to obtain optimal solutions quickly. With this, we managed to create a Python program that uses this algorithm throughout the length of a full day, that outputs a planning to follow on each node of this network. We applied a simple model to this program, based on real life data of the Chilean electrical network that updates every 5 minutes. The results are presented and explained visually in the form of plots depending on time, with numerical recaps of the full day. 
\end{document}